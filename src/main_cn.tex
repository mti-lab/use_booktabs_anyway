\RequirePackage{plautopatch}  % From https://qiita.com/wtsnjp/items/76557b1598445a1fc9da

\documentclass{article}
\usepackage{xeCJK}
\usepackage{hyperref,graphicx}

% To fix garbled text in the table of contents
\usepackage{pxjahyper}

% Change the link style
% https://tex.stackexchange.com/questions/823/remove-ugly-borders-around-clickable-cross-references-and-hyperlinks
\hypersetup{
  colorlinks = true, % Colours links instead of ugly boxes
  urlcolor = blue, % Colour for external hyperlinks
  linkcolor = blue, % Colour of internal links
  citecolor = red % Colour of citations
}

\usepackage[left=1.5cm, right=1.5cm, top=2cm, bottom=3cm]{geometry}

% For beautiful source-code
% https://www.overleaf.com/learn/latex/Code_Highlighting_with_minted
\usepackage{minted}

% For text box (screen)
\usepackage{ascmac}

% For math stuff
\usepackage{amsmath}
\usepackage{amsfonts}
\usepackage{bm}

% For a beautiful table
\usepackage{booktabs}
\usepackage{multirow}

% Alias
\newcommand{\Tref}[1]{表\ref{#1}}
\newcommand{\Eref}[1]{式\ref{#1}}
\newcommand{\Fref}[1]{图\ref{#1}}
\newcommand{\Aref}[1]{アルゴリズム\ref{#1}}
\newcommand{\Sref}[1]{\ref{#1}章}

\renewcommand\tablename{表}
\renewcommand\figurename{图}

\newcommand{\todo}[1]{\textbf{\textcolor{cyan}{[\textsc{TODO:} #1]}}}

\usepackage[fontsize=9pt]{fontsize}
\usepackage{indentfirst}
\setlength{\parindent}{2em}
\linespread{1.2}

\begin{document}

\title{总而言之来使用 booktabs 吧}
\author{东京大学 情报理工学系研究科 讲师 松井勇佑}
\date{初版 2022年9月5日 / 更新日期 2023年1月23日}
\maketitle

\begin{screen}
    \begin{itemize}
    \setlength{\itemsep}{-0.5ex}
    \item 本资料的GitHub仓库: \url{https://github.com/matsui528/use_booktabs_anyway}
    \item 作者的主页:\url{http://yusukematsui.me}
    \end{itemize}
\end{screen}

\section{序言}
在论文中插入表格时,来使用booktabs的宏包吧。
booktabs是在一般Tex表格基础上,可对线和宽度进行适当调整的宏包。
仅通过使用booktabs就能轻松让表格变得整洁美观。
不仅如此,按照booktabs的思路(以行为本、不插入竖线等)来制作表格的话,会使制成的表格更加清晰易懂。以下是两篇基础文献。
\begin{itemize}
    \setlength{\itemsep}{-0.5ex}
    \item \href{http://mirrors.ctan.org/macros/latex/contrib/booktabs/booktabs.pdf}{官方文档}。细节的设定等请参照CTAN的官方文档。
    \item \href{https://people.inf.ethz.ch/markusp/teaching/guides/guide-tables.pdf}{Markus P\"{u}schel, ``Small Guide to Making Nice Tables''}. 与本资料同样是介绍booktabs的文章。
\end{itemize}
本文以上述文献和我的个人经验为基础,可能存在错误或更好的方法。如果有感想建议或者有想指出的错误的话,请务必在issue提出。


\section{基本使用方法}
请看下面两个表格。


\begin{table}[h]
    \begin{minipage}{0.48\linewidth}
        \centering
        \begin{tabular}{|c|c|c|} \hline
            姓名 & 身高 (cm) & 体重 (kg) \\ \hline
            奥迪罗·汉宁古  & 173 & 62.2 \\ \hline 
            奥利法·伊诺埃  & 178 & 69.5 \\ \hline   
        \end{tabular}
        \caption{普通表格}
        \label{tbl:bad_example}
    \end{minipage}
    \hfill
    \begin{minipage}{0.48\linewidth}
        \centering
        \begin{tabular}{@{}lll@{}} \toprule
            姓名 & 身高 (cm) & 体重 (kg) \\ \midrule
            奥迪罗·汉宁古  & 173 & 62.2 \\ 
            奥利法·伊诺埃  & 178 & 69.5 \\ \bottomrule   
        \end{tabular}
        \caption{使用booktabs的表格}
        \label{tbl:good_example}
    \end{minipage}
\end{table}


左边是使用Tex的一般功能制作出的表格,估计大家最开始做的表格都是这种感觉的吧。
右边是使用booktabs对相同内容进行整理后的例子。是不是变得整洁又大方了呢?
左侧表格的tex代码如下所示。
\begin{minted}[mathescape, % 数式使える
    linenos, % 行番号
    fontfamily=courier, % いい感じのフォント
    breaklines, % 長すぎる行をうまく改行
    %breakindent=20pt, % 改行後の左からの幅
    fontsize=\small, % フォントサイズ
    %numbersep=2pt, % 行番号の内側よせ余白
    frame=single, % lines:上下に線 single:線で囲う
    %baselinestretch=0.7, % 行間
    ]{tex}
% “表1 普通表格”
\begin{tabular}{|c|c|c|} \hline
    姓名 & 身高 (cm) & 体重 (kg) \\ \hline
    奥迪罗·汉宁古  & 173 & 62.2 \\ \hline 
    奥利法·伊诺埃  & 178 & 69.5 \\ \hline  
\end{tabular}
\end{minted}
与之相对地,右侧使用booktabs的表格的代码如下所示。
\begin{minted}[mathescape, % 数式使える
    linenos, % 行番号
    fontfamily=courier, % いい感じのフォント
    breaklines, % 長すぎる行をうまく改行
    %breakindent=20pt, % 改行後の左からの幅
    fontsize=\small, % フォントサイズ
    %numbersep=2pt, % 行番号の内側よせ余白
    frame=single, % lines:上下に線 single:線で囲う
    %baselinestretch=0.7, % 行間
    ]{tex}
% “表2 使用booktabs的表格”
\begin{tabular}{@{}lll@{}} \toprule
    姓名 & 身高 (cm) & 体重 (kg) \\ \midrule
    奥迪罗·汉宁古  & 173 & 62.2 \\ 
    奥利法·伊诺埃  & 178 & 69.5 \\ \bottomrule  
\end{tabular}
\end{minted}
接下来将介绍将左边表格变换成右边表格的机械化规则。按照以下步骤来做就可以实现了。
\begin{enumerate}
    \setlength{\itemsep}{-0.5ex}
    \item 导入booktabs的宏包:在代码起始位置添加\mintinline{tex}{\usepackage{booktabs}}来导入宏包。忘记这一步的话肯定是无法顺利进行下去的。
    \item 去除竖线:\Tref{tbl:bad_example}里的列的设定为 \mintinline{tex}{{|c|c|c|}}。要通过修改这个地方来把竖线去掉,也就是说,将它改为 \mintinline{tex}{{ccc}}。表格里基本上是不需要竖线的。无论如何都需要竖线时,大多数情况下把表格进行分割会更好。
    \item 设置为左对齐:也就是把 \mintinline{tex}{{ccc}} 改为 \mintinline{tex}{{lll}}。推荐首先全部进行左对齐,如果觉得不够美观的话再考虑居中或者右对齐。
    \item 添加神奇的间隔符:把神奇的间隔符 \mintinline{tex}{@{}} 添加到列的设置的两端。
    即把 \mintinline{tex}{{lll}} 改成 \mintinline{tex}{{@{}lll@{}}}。很多人容易忘记这一步。
    添加了这个间隔符的话,可以去掉表格两端多余的空白,使表格看上去更加美观。
    未添加间隔符的不够整洁的表格如\Tref{tbl:wo_space}所示,添加了间隔符后变美观的表格如\Tref{tbl:w_space}所示。
    \item 删除横线并改成rule符号:把横线\mintinline{tex}{\hline}全部删除,“上框线”用\mintinline{tex}{\toprule}、
    “表头后”用\mintinline{tex}{\midrule}、“下框线”用\mintinline{tex}{\bottomrule}。
\end{enumerate}
这样就完成了。只是通过增加或删除操作就能带来截然不同的视觉效果。到这里本资料所要传达的内容已经实现了八成。接下来是扩展内容的介绍。

\begin{table}[h]
    \begin{minipage}{0.48\linewidth}
        \centering
        \begin{tabular}{lll} \toprule
            姓名 & 身高 (cm) & 体重 (kg) \\ \midrule
            奥迪罗·汉宁古  & 173 & 62.2 \\ 
            奥利法·伊诺埃  & 178 & 69.5 \\ \bottomrule  
        \end{tabular}
        \caption{未使用\texttt{@\{\}}的不够整洁的表格}
        \label{tbl:wo_space}
    \end{minipage}
    \hfill
    \begin{minipage}{0.48\linewidth}
        \centering
        \begin{tabular}{@{}lll@{}} \toprule
            姓名 & 身高 (cm) & 体重 (kg) \\ \midrule
            奥迪罗·汉宁古  & 173 & 62.2 \\ 
            奥利法·伊诺埃  & 178 & 69.5 \\ \bottomrule  
        \end{tabular}
        \caption{使用\texttt{@\{\}}后变美观的表格}
        \label{tbl:w_space}
    \end{minipage}
\end{table}


\section{扩展篇}
扩展篇将介绍制作优秀表格的方针和各种技术。

\subsection{以行为本}
建议时刻谨记\textbf{以行为一个单位},表格是\textbf{行的集合}这件事。
某次测量或某条数据就是一行,各行之间是独立的。
其次,每一行在概念上都是“同一级别的东西”。用编程语言来形容的话就是“同一类型的实例”。
如果一个表格能够只由“表头(类型定义)”和“行的集合(实例的集合)”构成,那它就是一个易于理解的表格。

在\Tref{tbl:good_example}的例子中,第一行的表头部分是“姓名”等对列的说明,第二行以后的主体部分满足“一个人的数据为一行”原则。
在这个例子中,即使改变每行的排列顺序也没有问题(交换奥迪罗和奥利法的顺序也没关系)。
虽然在实际制作表格时,以身高为基准进行了排序,但从含义上来讲,交换各行的顺序也没有关系。我们的目标就是能够制作出这样的表格。
如果改变表格主体部分的排列顺序会感到不自然的话,重新考虑表格的构成或者考虑能否用图表来表示会比较好。

booktabs的三种横线(\mintinline{tex}{\toprule}, \mintinline{tex}{\midrule}, \mintinline{tex}{\bottomrule})与这种以行为单位的表格的构成是相匹配的。
最上面一条线是\mintinline{tex}{\toprule},最下面一条线是\mintinline{tex}{\bottomrule},然后就是在“表头”和“行的集合”之间插入\mintinline{tex}{\midrule}。这种是基本的格式。

当然上述只是原则上的格式,也存在例外情况。比如将“合计”或“平均”那行放到最初或最后(\Tref{tbl:avg1})。
由于合计或平均与其他行相关,所以它们不是独立于其他行的。很多情况下,会将合计或平均放到表格的最初或最后。
在这个例子中为了便于理解,也可以像\Tref{tbl:avg2}那样在“平均”之上画一条横线。只不过需要注意不要画太多的横线。

在制作表格时,除了遵守“以行为本”的原则外,还要在制作过程中逐渐摸索怎样才能更加便于读者理解,以及怎样才能充分利用有限的纸面空间。


\begin{table}[h]
    \begin{minipage}{0.48\linewidth}
        \centering
        \begin{tabular}{@{}lll@{}} \toprule
            姓名 & 身高 (cm) & 体重 (kg) \\ \midrule
            奥迪罗·汉宁古  & 173 & 62.2 \\ 
            奥利法·伊诺埃  & 178 & 69.5 \\ 
            奥伊·宁格      & 167 & 52.1 \\ 
            平均               & 172.7  & 61.3 \\ \bottomrule   
        \end{tabular}
        \caption{把平均放在最后的例子}
        \label{tbl:avg1}
    \end{minipage}
    \hfill
    \begin{minipage}{0.48\linewidth}
        \centering
        \begin{tabular}{@{}lll@{}} \toprule
            姓名 & 身高 (cm) & 体重 (kg) \\ \midrule
            奥迪罗·汉宁古  & 173 & 62.2 \\ 
            奥利法·伊诺埃  & 178 & 69.5 \\ 
            奥伊·宁格      & 167 & 52.1 \\ 
            \midrule
            平均               & 172.7  & 61.3 \\ \bottomrule   
        \end{tabular}
        \caption{把平均放在最后并在其上面加了横线的例子}
        \label{tbl:avg2}
    \end{minipage}
    \hfill
\end{table}


\subsection{行的分组}
让表格变得直观的方法之一就是行的分组。
\Tref{tbl:row_group1}是原本的表格,分别展示了Hoge的方法和我们的方法下的速度和精度。此外,每个方法还有一个参数$k$,表中也展示了$k$的值的变化。这个表格是符合“一行是一个对象”原则的以行为本的表格。

另一方面,最左列的Hoge和Ours并列在一起,对于读者来说存在信息冗余。
这种情况下,像\Tref{tbl:row_group2}这样,将冗余的项目删除,对行进行分组会更好。
这种方式可以在保证表格信息量的前提下达到缩减字数的效果。

实现\Tref{tbl:row_group2}这种形式就已经可以了。此外,还可以在Hoge和Ours中间用\mintinline{tex}{\multirow}将它们区分开,如\Tref{tbl:row_group3}所示。在这个例子里,由于没有横线的话会比较难懂,所以用\mintinline{tex}{\midrule}添加了一条横线。
\Tref{tbl:row_group2}和\Tref{tbl:row_group3}哪个更好,主要取决于实际情况。
如果一个表格不加很多横线就会变得很难懂的话,把它分割成几个表格可能会更加便于理解。


\begin{table}[h]
    \begin{minipage}[t]{0.32\linewidth}
        \centering
        \begin{tabular}{@{}llll@{}} \toprule
            方法 & $k$ & 速度 (ms) & 精度 (Recall) \\ \midrule
            Hoge  & 16 & 0.32 & 0.21  \\ 
            Hoge  & 32 & 0.61 & 0.44  \\ 
            Ours  &  9 & 0.47 & 0.26 \\ 
            Ours  & 18 & 0.99 & 0.77 \\ \bottomrule   
        \end{tabular}
        \caption{原本的表格}
        \label{tbl:row_group1}
    \end{minipage}
    \hfill
    \begin{minipage}[t]{0.32\linewidth}
        \centering
        \begin{tabular}{@{}llll@{}} \toprule
            方法 & $k$ & 速度 (ms) & 精度 (Recall) \\ \midrule
            Hoge  & 16 & 0.32 & 0.21  \\ 
                  & 32 & 0.61 & 0.44  \\ 
            Ours  &  9 & 0.47 & 0.26 \\ 
                  & 18 & 0.99 & 0.77 \\ \bottomrule   
        \end{tabular}
        \caption{将行进行分组后的表格。这样就可以。}
        \label{tbl:row_group2}
    \end{minipage}
    \hfill
    \begin{minipage}[t]{0.32\linewidth}
        \centering
        \begin{tabular}{@{}llll@{}} \toprule
            方法 & $k$ & 速度 (ms) & 精度 (Recall) \\ \midrule
            \multirow{2}{*}{Hoge}  & 16 & 0.32 & 0.21  \\ 
                                   & 32 & 0.61 & 0.44  \\ \midrule  
            \multirow{2}{*}{Ours}  &  9 & 0.47 & 0.26 \\ 
                                   & 18 & 0.99 & 0.77 \\ \bottomrule   
        \end{tabular}
        \caption{为了更加集中额外加了一条横线。这样也可以。要注意不要加太多横线。}
        \label{tbl:row_group3}
    \end{minipage}
\end{table}

此外,可以进行多次行的分组和合并,如\Tref{tbl:row_group4}所示。
这是大学附近拉面店的整合信息。整体思路是先对第一列的项目进行排序然后去除冗余,再对第二列项目进行排序然后去除冗余,如此反复。
请注意如果过度合并的话可能会变得难以阅读。

\begin{table}[h]
    \centering
    \begin{tabular}{@{}llll@{}} \toprule
        最近的车站 & 店铺名称 & 菜品 & 价格(元) \\ \midrule
        本乡三丁目 & 海手 & natsu拉面 & 700 \\
                  &     & 拌面X & 850 \\
                  & IBASA & 拉面 & 700 \\
                  &       & 冷面 & 650 \\
        東大前  & 用心面 & 拉面 & 700  \\ 
              &         & 蘸面 & 800  \\ \bottomrule   
    \end{tabular}
    \caption{进行了两次行的分组合并的例子}
    \label{tbl:row_group4}
\end{table}


\subsection{行的分级}
接下来介绍行的分级技巧。
\Tref{tbl:row_hierarchy1}中总结了店里售卖的商品信息。每行之间相互独立,是一个优秀的表格。
由于这里存在类别信息的冗余,所以使用“行的分组”会变得更加整洁,像\Tref{tbl:row_hierarchy2}这样就可以。

在此基础上,利用类别信息对商品进行分级的例子如\Tref{tbl:row_hierarchy3}所示。
在这里删掉了类别那一列,把它放到了各个商品的上一级。
像这样通过删除列来减少字数,以达成行的分级。这里只通过插入空白就实现了字符的缩进,具体请看Tex的代码。


\begin{table}[b]
    \begin{minipage}[t]{0.32\linewidth}
        \centering
        \begin{tabular}{@{}llll@{}} \toprule
            商品 & 类别 & 价格(元)& 位置(层) \\ \midrule
            猪肉  & 肉 & 300 & 2  \\ 
            牛肉  & 肉 & 500 & 2  \\ 
            番茄  & 蔬菜 & 100 & 3 \\ 
            黄瓜  & 蔬菜 & 200 & 3 \\ 
            卷心菜  & 蔬菜 & 30 & 4 \\ \bottomrule   
        \end{tabular}
        \caption{原本的表格}
        \label{tbl:row_hierarchy1}
    \end{minipage}
    \hfill
    \begin{minipage}[t]{0.32\linewidth}
        \centering
        \begin{tabular}{@{}llll@{}} \toprule
            商品 & 类别 & 价格(元)& 位置(层) \\ \midrule
            猪肉  & 肉 & 300 & 2  \\ 
            牛肉  &  & 500 & 2  \\ 
            番茄  & 蔬菜 & 100 & 3 \\ 
            黄瓜  &  & 200 & 3 \\ 
            卷心菜  & & 30 & 4 \\ \bottomrule   
        \end{tabular}
        \caption{根据类别分组后的例子}
        \label{tbl:row_hierarchy2}
    \end{minipage}
    \hfill
    \begin{minipage}[t]{0.32\linewidth}
        \centering
        \begin{tabular}{@{}lll@{}} \toprule
            商品 & 价格(元)& 位置(层) \\ \midrule
            肉       & & \\
            ~~猪肉     & 300 & 2  \\ 
            ~~牛肉     & 500 & 2  \\ 
            蔬菜     & & \\
            ~~番茄   & 100 & 3 \\ 
            ~~黄瓜 & 200 & 3 \\ 
            ~~卷心菜 & 30 & 4 \\ \bottomrule   
        \end{tabular}
        \caption{进一步分级后的例子}
        \label{tbl:row_hierarchy3}
    \end{minipage}
\end{table}


\newpage
接下来介绍一个分级可以带来显著效果的例子。假设我们想要调查神奈川县各个市的经济规模。
\Tref{tbl:row_hierarchy4}展示了神奈川县的统计信息和其中三个行政市的信息。
为了比较,还列出了石川县和富山县的信息。这个表格每行都能独立表示各个地域的信息,也是一个优秀的表格。
与此同时,表格中既有表示县整体情况的“总计”,也有“横滨市”这样单独表示一个市的情况。由于各行所表示的级别不同,所以直观上可能难以理解。

因此将行进行分级后的例子如\Tref{tbl:row_hierarchy5}所示。
这种方式使理解变得容易了许多。比方说,很容易能看出,即使横滨市是“市”,它的人口也是面积是它10倍的石川“县”的3倍多。

需要注意的是,进行行的分组和分级都只是为了便于观看。
也就是说,制作出的表格要能随时都能恢复成“每行是一条数据”的“原本的表格”。


\begin{table}[h]
    \begin{minipage}[t]{0.48\linewidth}
        \centering
        \begin{tabular}{@{}llll@{}} \toprule
            县 & 市 & 面积(${km}^2$) & 人口(人) \\ \midrule
            神奈川县 & 总计 & 2,416 & 9,237,000 \\
            神奈川县 & 横滨市 & 437 & 3,774,000 \\
            神奈川县 & 川崎市 & 143 & 1,542,000 \\
            神奈川县 & 相模原市 & 328 & 726,000 \\
            石川县   & 总计 & 4,186 & 1,119,000 \\
            富山县   & 总计 & 4,247 & 1,018,000 \\ \bottomrule   
        \end{tabular}
        \caption{原本的表格}
        \label{tbl:row_hierarchy4}
    \end{minipage}
    \hfill
    \begin{minipage}[t]{0.48\linewidth}
        \centering
        \begin{tabular}{@{}lll@{}} \toprule
            地区 & 面积(${km}^2$) & 人口(人) \\ \midrule
            神奈川县 & & \\
            ~~~总计  & 2,416 & 9,237,000 \\
            ~~~横滨市 & 437 & 3,774,000 \\
            ~~~川崎市 & 143 & 1,542,000 \\
            ~~~相模原市 & 328 & 726,000 \\
            石川县   & 4,186 & 1,119,000 \\
            富山县   & 4,247 & 1,018,000 \\ \bottomrule   
        \end{tabular}
        \caption{将行进行分级后的例子}
        \label{tbl:row_hierarchy5}
    \end{minipage}
\end{table}


\subsection{列的分级}
接下来介绍“列的分级”。由于每列表示的都是完全不同的东西,所以无法进行列的分组。不过可以进行列的分级,其中一个例子如\Tref{tbl:col_hierarchy1}所示。
在这个表中展示了不同方法的最小误差、平均误差和最大误差。
虽然就这样也没有问题,但每次都要写上“误差”很繁琐。
像这种情况,可以对表头的项目进行分级。

分级之后的例子如\Tref{tbl:col_hierarchy2}所示。在这里通过添加了一条横线,实现了对原本独立各列的整合。
这里写着“误差”的部分跨越了多个列,可用\mintinline{tex}{\multicolumn}命令来实现。
此外,“局部横线”可以通过\mintinline{tex}{\cmidrule}命令来实现。


\begin{table}[h]
    \begin{minipage}[t]{0.48\linewidth}
        \centering
        \begin{tabular}{@{}llll@{}} \toprule
            方法 & 最小误差 & 平均误差 & 最大误差 \\ \midrule
            Isomap & 0.23 & 0.44 & 0.92 \\
            LLE    & 0.10 & 0.73 & 1.82 \\ \bottomrule
        \end{tabular}
        \caption{原本的表格}
        \label{tbl:col_hierarchy1}
    \end{minipage}
    \hfill
    \begin{minipage}[t]{0.48\linewidth}
        \centering
        \begin{tabular}{@{}llll@{}} \toprule
            & \multicolumn{3}{c}{误差} \\ \cmidrule(l){2-4}
            方法 & 最小 & 平均 & 最大     \\ \midrule
            Isomap & 0.23 & 0.44 & 0.92 \\
            LLE    & 0.10 & 0.73 & 1.82 \\ \bottomrule
        \end{tabular}
        \caption{将列进行分级后的例子}
        \label{tbl:col_hierarchy2}
    \end{minipage}
\end{table}


“列的分级”同样也可以进行多次,其例子如\Tref{tbl:col_hierarchy3}所示。这里表示了各个县的各种信息。
将气温和降水量进行分级后,可以使表格变得更加易于观看。同时,由于将单位的声明都写到了上面,所以使表格在横向上缩短了很多,使得没有分级的项目(县名和人口密度)也能被放到同一个表格里了。


\begin{table}[h]
    \centering
    \begin{tabular}{@{}lllllllll@{}} \toprule
        & \multicolumn{3}{c}{气温($\mathrm{^\circ C}$)} & \multicolumn{2}{c}{降水量(mm)} & \multicolumn{2}{c}{交通方式} & \\ \cmidrule(lr){2-4} \cmidrule(lr){5-6} \cmidrule(lr){7-8}
        县名 & 最高 & 平均 & 最低 & 8月 & 12月 & 新干线 & 飞机 & 人口密度(人/${km}^2$) \\ \midrule
        石川县 & 32 & 20 & -1 & 179.8 & 304.7 & \checkmark & \checkmark & 267 \\
        静冈县 & 27 & 23 & 5 & 250.9 & 63.0 & \checkmark & \checkmark & 461 \\
        冲绳县 & 33 & 28 & 15 & 175.4 & 104.4 &  & \checkmark & 643 \\ \bottomrule
    \end{tabular}
    \caption{对列进行多次分级的表格}
    \label{tbl:col_hierarchy3}
\end{table}


\newpage
\subsection{局部横线:cmidrule}
在这里来介绍一下在“列的分级”里提过的\mintinline{tex}{\cmidrule}吧。\Tref{tbl:col_hierarchy2}的代码如下所示。

\begin{minted}[mathescape, % 数式使える
    linenos, % 行番号
    fontfamily=courier, % いい感じのフォント
    breaklines, % 長すぎる行をうまく改行
    %breakindent=20pt, % 改行後の左からの幅
    fontsize=\small, % フォントサイズ
    %numbersep=2pt, % 行番号の内側よせ余白
    frame=single, % lines:上下に線 single:線で囲う
    %baselinestretch=0.7, % 行間
    ]{tex}
\begin{tabular}{@{}llll@{}} \toprule
           & \multicolumn{3}{c}{误差} \\ \cmidrule(l){2-4}
    方法   & 最小 & 平均 & 最大           \\ \midrule
    Isomap & 0.23 & 0.44 & 0.92         \\
    LLE    & 0.10 & 0.73 & 1.82         \\ \bottomrule
\end{tabular}
\end{minted}

\mintinline{tex}{\cmidrule(l){2-4}}中的“\texttt{l}”指的是是否删掉横线的两端。\texttt{l}表示去掉左侧,\texttt{r}表示去掉右侧,\texttt{lr}表示将两端都去掉。
通过这种细微的删减,来调整表格的外观。
具体例子如\Tref{tbl:cmidrule1}到\Tref{tbl:cmidrule4}所示。虽然只有细小的差别,但知道的话会很方便。
\texttt{\{2-4\}}指的是对应的列数。通过不断改变设置,来对表格进行调整就可以。


\begin{table}[h]
    \begin{minipage}[t]{0.24\linewidth}
        \centering
        \begin{tabular}{@{}llll@{}} \toprule
            & \multicolumn{3}{c}{误差} \\ \cmidrule(){2-4}
            方法 & 最小 & 平均 & 最大     \\ \midrule
            Isomap & 0.23 & 0.44 & 0.92 \\
            LLE    & 0.10 & 0.73 & 1.82 \\ \bottomrule
        \end{tabular}
        \caption{\texttt{cmidrule()\{2-4\}}}
        \label{tbl:cmidrule1}
    \end{minipage}
    \hfill
    \begin{minipage}[t]{0.24\linewidth}
        \centering
        \begin{tabular}{@{}llll@{}} \toprule
            & \multicolumn{3}{c}{误差} \\ \cmidrule(l){2-4}
            方法 & 最小 & 平均 & 最大     \\ \midrule
            Isomap & 0.23 & 0.44 & 0.92 \\
            LLE    & 0.10 & 0.73 & 1.82 \\ \bottomrule
        \end{tabular}
        \caption{\texttt{cmidrule(\textcolor{red}{l})\{2-4\}}}
        \label{tbl:cmidrule2}
    \end{minipage}
    \hfill
    \begin{minipage}[t]{0.24\linewidth}
        \centering
        \begin{tabular}{@{}llll@{}} \toprule
            & \multicolumn{3}{c}{误差} \\ \cmidrule(r){2-4}
            方法 & 最小 & 平均 & 最大     \\ \midrule
            Isomap & 0.23 & 0.44 & 0.92 \\
            LLE    & 0.10 & 0.73 & 1.82 \\ \bottomrule
        \end{tabular}
        \caption{\texttt{cmidrule(\textcolor{blue}{r})\{2-4\}}}
        \label{tbl:cmidrule3}
    \end{minipage}
    \hfill
    \begin{minipage}[t]{0.24\linewidth}
        \centering
        \begin{tabular}{@{}llll@{}} \toprule
            & \multicolumn{3}{c}{误差} \\ \cmidrule(lr){2-4}
            方法 & 最小 & 平均 & 最大     \\ \midrule
            Isomap & 0.23 & 0.44 & 0.92 \\
            LLE    & 0.10 & 0.73 & 1.82 \\ \bottomrule
        \end{tabular}
        \caption{\texttt{cmidrule(\textcolor{red}{l}\textcolor{blue}{r})\{2-4\}}}
        \label{tbl:cmidrule4}
    \end{minipage}
\end{table}


顺便一提,\Tref{tbl:col_hierarchy3}是将\mintinline{tex}{\cmidurle}设置成了“\mintinline{tex}{\cmidrule(lr){2-4} \cmidrule(lr){5-6} \cmidrule(lr){7-8}}”,加了“lr”去掉了线的两端。如果不去掉两端的话就如\Tref{tbl:cmidrule5}所示。
这里删除了所有的“lr”,设置成了\mintinline{tex}{\cmidrule(){2-4}}的形式。
这样导致“气温”“降水量”“交通方式”之间的线都连到了一起,因此这里加入适当的空白会使表格更加便于理解。


\begin{table}[h]
    \centering
    \begin{tabular}{@{}lllllllll@{}} \toprule
        & \multicolumn{3}{c}{气温($\mathrm{^\circ C}$)} & \multicolumn{2}{c}{降水量(mm)} & \multicolumn{2}{c}{交通方式} & \\ \cmidrule(){2-4} \cmidrule(){5-6} \cmidrule(){7-8}
        县名 & 最高 & 平均 & 最低 & 8月 & 12月 & 新干线 & 飞机 & 人口密度(人/${km}^2$) \\ \midrule
        石川县 & 32 & 20 & -1 & 179.8 & 304.7 & \checkmark & \checkmark & 267 \\
        静冈县 & 27 & 23 & 5 & 250.9 & 63.0 & \checkmark & \checkmark & 461 \\
        冲绳县 & 33 & 28 & 15 & 175.4 & 104.4 &  & \checkmark & 643 \\ \bottomrule
    \end{tabular}
    \caption{没在\Tref{tbl:col_hierarchy3}中设定cmidrule删减的结果}
    \label{tbl:cmidrule5}
\end{table}



\subsection{将列移到行}
有时将列的内容移到行会使表格变得更简洁。虽然我\textbf{不推荐}这种技巧,但是在实际中经常会遇到这种情况所以还是来介绍一下。
举个例子,\Tref{tbl:col2row1}是原本的表格。
这里在3个数据集上对2种方法进行了评价,并列出了运行时间。每行都有一个相应的结果,是一个优秀的表格。
对每个方法进行了行的分组。本来这样就可以,但是可能会有以下两个需求。
\begin{itemize}
    \setlength{\itemsep}{-0.5ex}
    \item 想在同一个Dataset里比较不同方法,但是数值不在一起。比如MNIST的k-means (10.2)和Ours (8.3)离得较远。
    \item 继续增加Method或Dataset的话,会使表格变得过于竖长。想在横向上增加长度。
\end{itemize}
遇到这种情况时该怎么办才好呢?

可以考虑\textbf{将某一列(在这里是Dataset)移到行}。“一条数据(一行)”不再是“方法・数据集・运行时间”,而是变成
“方法・数据集1的运行时间・数据集2的运行时间・数据集3的运行时间”。这样转换之后的表格就如\Tref{tbl:col2row2}所示。
这种方式同时满足了“将同一个数据集的数值放到相近位置”“增加横向长度”的两个要求。但相应地,表头的部分明显变得冗长了。

\begin{table}[h]
    \begin{minipage}{0.32\linewidth}
        \centering
        \begin{tabular}{@{}lll@{}} \toprule
            Method & Dataset & Runtime (ms) \\ \midrule
            k-means   & MNIST    & 10.2 \\ 
                      & ImageNet & 45.3 \\ 
                      & Places   & 57.1 \\ 
            Ours      & MNIST    & 8.3 \\ 
                      & ImageNet & 39.1 \\ 
                      & Places   & 82.3 \\ \bottomrule   
        \end{tabular}
        \caption{原本的表格}
        \label{tbl:col2row1}
    \end{minipage}
    \hfill
    \begin{minipage}{0.67\linewidth}
        \centering
        \begin{tabular}{@{}llll@{}} \toprule
                   & Runtime (ms) for & Runtime (ms) for & Runtime (ms) for         \\
            Method & Dataset=MNIST & Dataset=ImageNet & Dataset=Places \\ \midrule
            k-means   & 10.2 & 45.3 & 57.1 \\ 
            Ours      & 8.3  & 39.1 & 82.3 \\ \bottomrule   
        \end{tabular}
        \caption{将Dataset这一列移到行之后。表头变得冗余。}
        \label{tbl:col2row2}
    \end{minipage}
\end{table}

面对这种情况,使用“列的分级”后就变成了\Tref{tbl:col2row3}。
明显变得好多了,就这样结束也没有问题。
只不过这里“Dataset=”在表头出现了3次,显得冗余。
\Tref{tbl:col2row4}和\Tref{tbl:col2row5}是将表格变得更加简洁的2个方法。


\begin{table}[h]
    \centering
    \begin{tabular}{@{}llll@{}} \toprule
                & \multicolumn{3}{c}{Runtime (ms)}  \\ \cmidrule(l){2-4}
        Method & Dataset=MNIST & Dataset=ImageNet & Dataset=Places \\ \midrule
        k-means   & 10.2 & 45.3 & 57.1 \\ 
        Ours      & 8.3  & 39.1 & 82.3 \\ \bottomrule   
    \end{tabular}
    \caption{将列进行分级后的结果。虽然比之前变好了,但“Dataset=”的部分还是冗余。}
    \label{tbl:col2row3}
\end{table}


\Tref{tbl:col2row4}去掉了\Tref{tbl:col2row3}中“Dataset=”的部分,使表格变得更加容易观看了。在这个例子里,就这样也能让人明白表格的含义。然而,由于完全没写“Dataset=”的说明,所以读者可能会不明白“MNIST”的部分是什么意思。
因此,不另外说明Runtime下面项目的含义的话,这种写法是行不通的。


\Tref{tbl:col2row5}把\Tref{tbl:col2row3}中的“Runtmie (ms)”改成了“Dataset”,这是最简洁的写法。由于没有解释表格里数值的含义,所以有必要在注释里加上“这是Runtime (ms)的表格”的说明。
这就是“当表格由唯一一种数据构成时,有时可以通过把对数值的说明放到注释里的方式来减少表头信息量,使表格变得简洁”的方法。
顺便一提,这种只含唯一一种数据的表格,与本资料的其他表格相比有一个特点。
那就是,表头里不含对元素的说明。我个人不是很喜欢这种方式,我还是认为表头里包含元素说明比较好。
不过也可能存在由于空间限制必须要这么写的情况。


\begin{table}[h]
    \begin{minipage}{0.48\linewidth}
        \centering
        \begin{tabular}{@{}llll@{}} \toprule
                   & \multicolumn{3}{c}{Runtime (ms)}  \\ \cmidrule(l){2-4}
            Method & MNIST & ImageNet & Places \\ \midrule
            k-means   & 10.2 & 45.3 & 57.1 \\ 
            Ours      & 8.3  & 39.1 & 82.3 \\ \bottomrule   
        \end{tabular}
        \caption{省略了“Dataset=”的情况。虽然变简洁了,但是哪里都没有关于“Dataset=”的信息。}
        \label{tbl:col2row4}
    \end{minipage}
    \hfill
    \begin{minipage}{0.48\linewidth}
        \centering
        \begin{tabular}{@{}llll@{}} \toprule
                   & \multicolumn{3}{c}{Dataset}  \\ \cmidrule(l){2-4}
            Method & MNIST & ImageNet & Places \\ \midrule
            k-means   & 10.2 & 45.3 & 57.1 \\ 
            Ours      & 8.3  & 39.1 & 82.3 \\ \bottomrule   
        \end{tabular}
        \caption{省略了“Runtime (ms)”的情况。虽然最简洁,但是有必要在注释里加上“这是Runtime (ms)的表格”的说明。}
        \label{tbl:col2row5}
    \end{minipage}
\end{table}


回归正题,这次讨论了很多种表格的写法,是因为我们既想从Method角度也想从Dataset角度进行分析。
由于表格只由行和列构成,所以想要在横向和纵向上都进行分析的时候,就会变成\Tref{tbl:col2row4}和\Tref{tbl:col2row5}这样类似于矩阵的形式。
因此,这样就逐渐远离了“一行是一次观测”的原则。
这种情况下,可以考虑使用图表。将同样内容用图表形式来表现的话就如\Fref{fig:single_val}所示。
这样既能实现“某个数据集下不同方法比较”也能实现“不同数据集下某个方法比较”。
在这次的例子中,就算再增加数据集或方法,图表所占的空间也不会改变。此外,还可以增加误差条或者转换成箱线图的形式。
另一方面,使用图表的缺点是无法显示具体的数值。比如当需要和既存方法的数值进行比较时,使用图表就会显得不够方便。
建议通过考虑表格和图表的优缺点,来选择最有效的表示信息的方式。


\begin{figure}[h]
    \centering
    \includegraphics[width=0.5\linewidth]{script/single_val.pdf}
    \caption{把\Tref{tbl:col2row4}和\Tref{tbl:col2row5}转换成图表。}
    \label{fig:single_val}
\end{figure}


\subsection{对表格制作感到无从下手的话}
如果你对于表格制作感到无从下手的话,可以试着按照以下顺序来做。
\begin{enumerate}
    \setlength{\itemsep}{-0.5ex}
    \item 根据“一行是一条数据”的原则来分解表格。就算横向上很短也没关系。
    \item 对最重要的那一项反复进行“行的分组”。需要的话也可以进行“行的分级”。
    \item 需要的话进行“列的分级”。
    \item 如果还是感到不合适,可以进行“将列移到行”和“行的分级”。然后省略掉冗长的说明。
    \item 如果还是感到不合适,可以考虑使用图表。
\end{enumerate}


\section{案例分析}
接下来通过几个例子,来学习怎样能使表格变美观吧。首先请看\Tref{tbl:use_case1}、\Tref{tbl:use_case2}和\Tref{tbl:use_case3}。
这里显示的是改变参数时各个方法的精度。看上去它们全都反映了同样的内容,但其实这里面包含了常见的错误。

首先来看\Tref{tbl:use_case1}。它表示的是当改变参数$T$时,ResNet50和我们的方法的精度变化。
这个表格第一眼看上去好像很工整,但是仔细看就会发现,本应是表头的第一行里,并不完全是表头信息。比如ResNet50的上面本来应该是“Method”,然而这里却变成了$T$。此外,“0.32”上面的“1”也不能成为对数据的说明。因此,这不是一个由“表头 + 行的集合”所构成的表格。
另外,这个表格里也没有对要素的说明(即说明数据表示的是精度)。也就是说,不明白数字的含义,必须要添加注释才行。对于这种小表格而言,不用注释就能在表中把所有信息都展示出来的表格应该是最整洁的。

相信很多人会把同样的内容用\Tref{tbl:use_case2}这种形式来表示。左上角的格子里包含“斜线”和“横向纵向的信息”。
由于在小学和中学学过这种方法,所以相信很多人都会这么写。
然而这种方式也不满足“表头 + 行的集合”的形式。
此外,这种写法经常需要添加竖线。

可能还有人会写成\Tref{tbl:use_case3}的形式。它比\Tref{tbl:use_case1}稍好,也能让人明白“0.32”上面是“$T=1$”。然而还是不能让人明白数字的含义是什么。


\begin{table}[h]
    \begin{minipage}{0.32\linewidth}
        \centering
        \begin{tabular}{@{}llll@{}} \toprule
            $T$ & 1 & 4 & 16 \\ \midrule
            ResNet50  & 0.32 & 0.54 & 0.77 \\ 
            Ours      & 0.41 & 0.81 & 1.23 \\ \bottomrule   
        \end{tabular}
        \caption{常见例子1}
        \label{tbl:use_case1}
    \end{minipage}
    \hfill
    \begin{minipage}{0.32\linewidth}
        \centering
        \begin{tabular}{|l|lll|} \hline
            Method \textbackslash ~ $T$ & 1 & 4 & 16 \\ \hline
            ResNet50  & 0.32 & 0.54 & 0.77 \\ 
            Ours      & 0.41 & 0.81 & 1.23 \\ \hline
        \end{tabular}
        \caption{常见例子2}
        \label{tbl:use_case2}
    \end{minipage}
    \hfill
    \begin{minipage}{0.32\linewidth}
        \centering
        \begin{tabular}{@{}llll@{}} \toprule
            & $T=1$ & $T=4$ & $T=16$ \\ \midrule
            ResNet50  & 0.32 & 0.54 & 0.77 \\ 
            Ours      & 0.41 & 0.81 & 1.23 \\ \bottomrule   
        \end{tabular}
        \caption{常见例子3}
        \label{tbl:use_case3}
    \end{minipage}
\end{table}


那么,怎样才能将这些表格变好看呢?首先试着将表格分解至“一行是一条数据”。
这里所谓的“一条数据”指的是“使用某个方法和某个参数$T$时,测量结果的精度(mAP)”。
结果如\Tref{tbl:use_case4}所示。这里在表头注明了各个要素是“精度”。虽然这样就可以了,但还是再来进一步思考一下吧。
实际上这个表格和\Tref{tbl:col2row1}的结构是相同的。因此进行“将$T$这一列移到行”和“列的分级”,整合后得到\Tref{tbl:use_case5}。这就相当于把\Tref{tbl:col2row1}转换成\Tref{tbl:col2row3}的过程。
由于这里$T=1$的描述已经很短了,所以这个表格已经可以最直观地展示信息了。
与\Tref{tbl:use_case3}相比,“mAP”的信息出现在了表格里,并且也增加了“Method”的说明。


\begin{table}[h]
    \begin{minipage}{0.48\linewidth}
        \centering
        \begin{tabular}{@{}lll@{}} \toprule
            Method & $T$ & mAP \\ \midrule
            ResNet50 & 1  & 0.32 \\
                    & 4  & 0.54 \\
                    & 16 & 0.77 \\
            Ours     & 1  & 0.41 \\
                    & 4  & 0.81 \\
                    & 16 & 1.23 \\ \bottomrule   
        \end{tabular}
        \caption{分解至“一行是一条数据”后的结果}
        \label{tbl:use_case4}
    \end{minipage}
    \hfill
    \begin{minipage}{0.48\linewidth}
        \centering
        \begin{tabular}{@{}llll@{}} \toprule
            & \multicolumn{3}{c}{mAP} \\ \cmidrule(l){2-4}
            Method & $T=1$ & $T=4$ & $T=16$ \\ \midrule
            ResNet50  & 0.32 & 0.54 & 0.77 \\ 
            Ours      & 0.41 & 0.81 & 1.23 \\ \bottomrule   
        \end{tabular}
        \caption{把$T$那一列移到行并进行列的分级后的结果。在这个例子中这种形式是最直观的。}
        \label{tbl:use_case5}
    \end{minipage}
\end{table}


不过,能将表格进一步省略成\Tref{tbl:col2row4}或\Tref{tbl:col2row5}的形式吗?结果如\Tref{tbl:use_case6}和\Tref{tbl:use_case7}所示。
\Tref{tbl:use_case6}里省略了“$T=$”的部分。在这个例子中,这个省略是行不通的,因为会让人不明白1, 4, 16是什么。所以还是不要这么写比较好。
\Tref{tbl:use_case7}里省略了“mAP”。这种情况下,需要在注释里加上对它的说明。在这个例子里似乎没有做这种省略的必要。


\begin{table}[h]
    \begin{minipage}{0.48\linewidth}
        \centering
        \begin{tabular}{@{}llll@{}} \toprule
            & \multicolumn{3}{c}{mAP} \\ \cmidrule(l){2-4}
            Method & 1 & 4 & 16 \\ \midrule
            ResNet50  & 0.32 & 0.54 & 0.77 \\ 
            Ours      & 0.41 & 0.81 & 1.23 \\ \bottomrule   
        \end{tabular}
        \caption{省略了“$T=$”的情况。这里会让人不明白1, 4, 16的含义,不是一个优秀的表格。}
        \label{tbl:use_case6}
    \end{minipage}
    \hfill
    \begin{minipage}{0.48\linewidth}
        \centering
        \begin{tabular}{@{}llll@{}} \toprule
            & \multicolumn{3}{c}{$T$} \\ \cmidrule(l){2-4}
            Method & 1 & 4 & 16 \\ \midrule
            ResNet50  & 0.32 & 0.54 & 0.77 \\ 
            Ours      & 0.41 & 0.81 & 1.23 \\ \bottomrule   
        \end{tabular}
        \caption{省略了“mAP”的情况。需要在注释里加上“这是mAP的表格”。}
        \label{tbl:use_case7}
    \end{minipage}
\end{table}


此外,这次的例子还可以用图表来表示,如\Fref{fig:use_case}所示。由于在这里$T$是可连续变化的变量、因此使用了折线图而不是柱状图。
请根据情况来决定图表和表格哪一个更好。


\begin{figure}[h]
    \centering
    \includegraphics[width=0.5\linewidth]{script/use_case.pdf}
    \caption{把\Tref{tbl:use_case5}转换成图表。}
    \label{fig:use_case}
\end{figure}


接下来,再来看一个例子吧。请看\Tref{tbl:use_case8}。
这个例子与\Tref{tbl:use_case1}类似,看上去好像很工整,但第一行不能被称之为表头,因此不能被称为一个优秀的表格。
并且不同于\Tref{tbl:use_case1}的第一列能用“Method”来概括,这个例子里找不到能够概括的单词。这要怎么办才好呢?
其实仔细观察这个例子就会发现,把行和列进行交换后的\Tref{tbl:use_case9}更加整洁,符合“每一行是一条数据”的原则。因此,从以行为本的角度来说,\Tref{tbl:use_case8}是更加优秀的表格。
然而,这里表头的项目太长,使得表格里存在较多空白。
怎样才能有效利用空白就需要各自的技术了。比如可以把单位放到下一行等。

\begin{table}[h]
    \begin{minipage}{0.48\linewidth}
        \centering
        \begin{tabular}{@{}llll@{}} \toprule
            $T$ & 1 & 4 & 16 \\ \midrule
            Runtime (sec)       & 102 & 110 & 159 \\ 
            Memory (B)  & 100 & 200 & 300 \\ \bottomrule   
        \end{tabular}
        \caption{原本的表格}
        \label{tbl:use_case8}
    \end{minipage}
    \hfill
    \begin{minipage}{0.48\linewidth}
        \centering
        \begin{tabular}{@{}lll@{}} \toprule
            $T$ & Runtime (sec) & Memory (B) \\ \midrule 
            1 & 102 & 100 \\
            4 & 110 & 200 \\
            16 & 159 & 300 \\ \bottomrule 
        \end{tabular}
        \caption{把行和列进行交换后的表格}
        \label{tbl:use_case9}
    \end{minipage}
\end{table}





\end{document}  
